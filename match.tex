\section{Match Pair Approximation}

\algoref{algo:match} approximates the match pairs in a specific section that consists of a sequence of receives in a common process $p$ and multiple sequences of sends from the potential sources to $p$.
The algorithm is the part of the function $\mathrm{MATCHAPPROX}$ in \algoref{algo:main} and is inspired by the match pair generation algorithm in the prior work \cite{DBLP:conf/kbse/HuangMM13}. 
Intuitively, the algorithm checks all the potential match pairs for the receives and sends in the section, and prunes obvious matches that cannot exist in any runtime implementation of the specification.

\begin{algorithm}
\caption{Match Approximate}\label{algo:match}
\begin{algorithmic}[1]
\For{$i\gets 1$ to $N_K$}
\State let $p_r$ be the source of the receive $R(p)_i$
\For{$j\gets 1$ to $\mathit{n_s}(p_s,p)$ for each possible sender $p_s$}
\If{$(p_r = \ast\vee p_r = p_s)\wedge i \geq j\wedge i \leq j + (N_K - \mathit{n_s}(p_s,p))$}
\State $M\gets M\cup\{\langle R(p)_i,S(p_s,p)_j \rangle\}$
\EndIf
\EndFor
\EndFor
\end{algorithmic}
\end{algorithm}

The algorithm first checks each receive $R(p)_i$ in a section at line 2. The subscript $i$ is the rank of $R(p)_i$. The process $p_r$ is the source of $R(p)_i$.
For each receive, the algorithm checks the sends with the ranks from $1$ to $n_s(p_s,p)$ in the sequence $S(p_s,p)$ for each potential source $p_s$ at line 3.
The subscript $j$ is the rank of the send $S(p_s,p)_j$. 

The condition at line 4 soundly prunes the match pair $\langle R(p)_i,S(p_s,p)_j\rangle$ that may never occur in the runtime with three rules. The first rule constrains the match pair in that either $R(p)_i$ is a wildcard receive or $R(p)_i$ and $S(p_s,p)_j$ have an identical source. The second rule and the third rule assert that the messages from a common source are received in a FIFO order. In precise, the second rule constrains that the rank of $R(p)_i$ is greater or equal to the rank of $S(p_s,p)_j$, indicating that the early messages from $p_s$ must be received by the preceding receives in $p$. The third rule also constrains the rank of $R(p)_i$ and the rank of $S(p_s,p)_j$, but in a more complex structure of an inequation. It asserts that there are a sufficient number of sends to match the preceding receives in $p$. This number, $j + (N_K - \mathit{n_s}(p_s,p))$, is estimated by considering all the potential sends including $j$ preceding sends from $p_s$ to $p$ and all the sends from other sources to $p$.
If the condition is satisfied, the match pair is added to the set $M$ at line 5. Note that \algoref{algo:match} is not complete meaning that the unprecise match pairs may be added to $M$.

\begin{lemma}
\algoref{algo:match} over-approximates the match pairs for all the receives and sends in a specific section; that is saying, that all the precise match pairs (and maybe some unprecise match pairs) in the section are added to the output $M$.
\label{lemma:match}
\end{lemma}
\begin{proof}
\algoref{algo:match} must never claim to be able to prune the precise match pairs. The algorithm considers all the pairs for a send and a receive in a specific section. Also, the condition at line 4 only prunes the pairs that may never occur in the runtime: the first rule validates the endpoints consistent for the send and the receive; the second and third rules validate the FIFO order in message delivery according to the semantics. Therefore, only unprecise match pairs are pruned. $\Box$
\end{proof}


%The condition at line 4 is then used to soundly prune the pair of $R(p)_i$ and $S(p_s,p)_j$ if they never match in the runtime with three rules. The first rule indicates that $R(p)_i$ is either a wildcard receive (the receive that may match a send from any source) or the source of $R(p)_i$ matches the source of $S(p_s,p)_j$. The second rule indicates that $i$ is greater or equal to $j$ meaning that the sends preceding $S(p_s,p)_j$ in $S(p_s,p)$ have to be matched with the receives in $p$. 