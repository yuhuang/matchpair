 \section{Introduction}
%Asynchronous message passing is a prevalent programming model in high performance computing (HPC). The model consists of two operations, send and receive, that are essential to message communication. The message communication is complicated because of the non-determinism that a receive may be matched with more than one send in the runtime. The use of match pair, a pair of a send and a receive that may potentially match in the runtime, is able to capture the message communication.
%The semantics can also be complicated because of two buffering settings in the runtime, infinite buffer (messages are buffered in the system) and zero buffer (no buffering in the system). Further, typical message passing standard such as message passing interface (MPI) specifies a special communication that uses collective operations to synchronize a program leading to a more complex program behavior. 
%Given the semantics, several common problems exist in message passing applications: the user-provided assertion may be violated in an execution; the program may deadlock for unexpected matching of receives; and the program may be incompatible with zero buffer semantics, meaning that no feasible schedule exists under zero buffer setting. 
Asynchronous message passing is a prevalent programming model in high performance computing (HPC).
The idea of message passing is simple at the outset, processes communicate by sending messages to one to another, but it does not take long to realize how that despite the simplicity of the model it still contains a lot of subtlety that affects program behavior. Depending on the message passing implementation, program behavior is affected by things such as message non-determinism (concurrent sends to a process can arrive in any order), buffering semantics in the runtime (a process may block if the buffer is full), or richer operations that message with groups of processes at the same time (collective operations). Given the inherent complexity of the message passing paradigm, determining if a message passing program is free of deadlock, compatible with a given runtime’s buffering semantics, or if the message non-determinism affects the correctness of the computation are all NP-complete problems \cite{DBLP:conf/kbse/HuangMM13,HuangNFM15,HuangDeadlock}. Showing any of these properties for any input program is hard.

Prior work on program correctness in the message passing paradigm can be roughly grouped into dynamic analysis where an existing runtime is manipulated to explore different scheduling outcomes \cite{DBLP:conf/ppopp/VakkalankaSGK08,DBLP:conf/sbmf/SharmaGB12}, model checking where a model of the original program is analyzed \cite{DBLP:conf/vmcai/Siegel07,DBLP:conf/pvm/Siegel07}, runtime verification where the program execution is observed but not manipulated \cite{DBLP:conf/sc/VetterS00,DBLP:conf/parco/KrammerBMR03,DBLP:conf/ptw/HilbrichSSM09}, and symbolic model checking where a model of the program is symbolically analyzed with an SMT solver \cite{DBLP:conf/kbse/HuangMM13,HuangNFM15,HuangDeadlock}. The research presented in this paper looks specifically at symbolic model checking of message passing programs.

The most efficient model encodings for symbolic model checking of message passing programs rely critically on the concept of a match pair. Efficient in this instance means those encodings that scale to program with a high degree of message non-determinism.  A match pair represents a send and receive pair that may be matched in some feasible execution of the program in the runtime. 
The crucial idea behind symbolic model checking of message passing programs is executing a message passing program to several traces. Each trace is constructed as a concurrent trace program (CTP) where the processes are statically known; and the sequential order of each process is maintained. Also, only the operations in a single path of the original program exist in the CTP where the path conditions are added to a set of assume operations to constrain the same path. This paper also considers CTPs in the discussion. The over-approximated match pairs for all the sends and receives in each CTP (including all the precise match pairs and maybe a few match pairs that may never occur in the runtime) are generated as input to the analysis. The work then encodes the CTP into an SMT problem that uses a set of formulas to constrain the program behavior based on the semantics. The typical errors are then checked by solving the SMT problem. If a satisfying assignment exists for the SMT problem, then the error is detected for a feasible schedule with a resolution of the match pairs to capture the message communication in the schedule. If the SMT problem is unsatisfiable, the CTP is free of that error because no feasible schedule exists with any resolution of the match pairs. 

An important aspect that impacts the performance of the SMT encoding is the size of the input match pairs. The encoding can be resolved much slowly for a large set of match pairs because there are exponential ways to resolve these match pairs. As such, a straightforward idea to improve the performance is to reduce the size of the match pairs. However, a naive way of reducing the match pairs such as randomly selecting a subset of the match pairs is not sufficient. The input match pairs must satisfy two properties: program completion (the program can run to completion assuming no deadlocks exist) and message non-determinism (the set has a certain degree of non-determinism such that a receive may be associated with more than one match pairs).  
%The algorithm in this paper explores a way to generate such a reduced set of match pairs.
To reduce the number of match pairs and thereby decrease the cost of verification, this paper describes an iterative algorithm to successively generate under-approximations of the true set of match pairs until all match pairs are generated. The under-approximations retain properties ``program completion" and ``message non-determinism". Also, the lack of some precise match pairs is not able to capture the full message communication space. Therefore, the SMT encoding with the under-approximated match pairs generated by the algorithm in this paper is only used to detect typical errors, but not to prove their absences.

Generally, the algorithm generates the match pairs in three steps. First, each process is sectioned, where a section contains a fixed number of sequential receives. 
%The number of the receives depends on three values that are statically known: the positive integer $k$ configured by the user, the number of all the potential sender processes and the number of the unmatched receives in the process. 
Match pairs for each section are then generated independently from other sections in the process. This independent sectioning effectively ignores combinations of match pairs, that are feasible, but only available when the considered concurrent outstanding sends are more than the number of receives in the section. As such, the second step of the algorithm is to add a sequence of sends from each sender process to a section in the destination process. The sends in each sequence are distributed sequentially from the sender process. The total number of the sends added to the section is equal to the number of receives in the same section. Finally, the algorithm generates the match pairs for the sends and receives in the same section by a list of simple rules based on ranks. A rank is a non-negative integer that represents the position of a send or a receive in a specific sequence in a section.


%The key insight of the solution is that each section is considered independently for match pair generation.  
%The algorithm does not generate the match pair for a send and a receive from two different sections, while in fact the send and the receive may match in the runtime. 
%As a result, the algorithm is able to under-approximate the precise match pairs as input to an SMT problem. 
%Due to the missed match pairs that are derived from matching sends and receives in different sections, the SMT problem may not encode the complete program behavior. However, if the problem is satisfiable, then it is known that a specific error exists for a feasible schedule that is resolved by the encoding. As such, the SMT encoding with a reduced set of match pairs can be used to detect errors. If the problem is, however, unsatisfiable for the under-approximated match pairs, it does not mean that there are provably no errors for any feasible execution in the runtime. It is then necessary to generate a larger set of match pairs by a new input K to capture more behavior for detection.


The paper includes the proof that the precise match pairs for any CTP can also be generated by bounding the positive integer $k$. Experiments further show that the new algorithm improves the prior work in the SMT encoding such that the runtime of error detection is drastically reduced and all the errors in the benchmarks are detected with the reduced match pairs.
The contributions include,
\begin{compactitem}
\item the efficient algorithm that under-approximates the precise match pairs for a CTP, 
%where the size depends on a positive integer $k$ configured by the user,
\item the proof that the precise match pairs for any CTP can be generated by the new algorithm with an appropriate positive integer $k$, and
\item the benchmarks that demonstrate the new algorithm improves the prior approach in the SMT encoding such that the runtime of error detection is drastically reduced and all the errors detected by the prior work are not missed. 
\end{compactitem}

The rest of the paper is organized as follows: 
Sections 2 presents the definition and semantics of CTP; Section 3 presents the general algorithm in the paper and the section matching in the algorithm; Section 4 presents the send distribution in the algorithm; Section 5 presents the match approximation in the algorithm; Section 6 gives the proof that the precise match pairs can be generated; Section 7 gives the experimental results; Section 8 discusses the related work; and Section 9 is the conclusion and future work.
