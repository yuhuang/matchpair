\section{Related Works}
There are several works related to match pairs.
Sharma et al. proposed the first push button model checker for MCAPI -- MCC \cite{DBLP:conf/fmcad/SharmaGMH09}. It indirectly controls the MCAPI runtime to verify MCAPI programs under zero buffer semantics. An obvious drawback of the work is its inability to analyze infinite buffer semantics which is known as a common runtime environment in message passing. A key insight, though, is the direct use of match pairs -- couplings for potential sends and receives.

A precise SMT encoding technique is proposed for detecting user-provided assertions for MCAPI programs \cite{DBLP:conf/kbse/HuangMM13}. The encoding is sound and complete and is easy to use to reason about infinite buffer semantics without requiring a precise match set. The work also provides an algorithm that runs in quadratic time complexity to generate a over-approximated match set based on the given execution trace. This approach is extended to checking zero buffer incompatibility for MPI semantics \cite{HuangNFM15}. 

Elwakil et al. proposed another encoding technique that is applied to MCAPI \cite{DBLP:conf/atva/ElwakilYW10,DBLP:conf/issta/ElwakilY10}. The encoding uses a non-obvious way to constrain the happens before relation in a program. The approach fails for two reasons. First, it does not find out all possible process interleavings under infinite buffering setting. Second, the foreknowledge assumption about the potential matches of send and receive operations does not apply for a large complex program execution. 

%A hybrid approach of static analysis and dynamic analysis also uses the SMT technique for detecting deadlocks in the pattern of orphaned receive \cite{deadlock-draft}. This approach first detects all possible potential deadlocks statically, then prunes the infeasible deadlocks by an abstracting machine with counting, and finally validate the deadlock for a feasible schedule. The validation require the SMT encoding for orphaned receive deadlock. 

Forejt et al. proposed a SAT based approach to detect deadlock in a single-path MPI program \cite{DBLP:conf/fm/ForejtKNS14}. The solution is correct and efficient for programs with a low degree of message non-determinism. However, since the size of the encoding is cubic, checking large programs is time consuming. The work also requires a match pair set.

There are other solutions for message passing program analysis.
The dynamic analyzer ISP implements the POE algorithm, a Dynamic Partial Order Reduction (DPOR) algorithm \cite{DBLP:conf/popl/FlanaganG05} applied to MPI programs \cite{DBLP:conf/ppopp/VakkalankaSGK08}. 
An extension is the MSPOE algorithm \cite{DBLP:conf/sbmf/SharmaGB12}. It operates by postponing the cooperative operations for message passing in transit until each process reaches a blocking call, and then determines the potential matches of send and receive operations in the runtime. 
%The solution is able to detect errors such as assertion violation and deadlock in an MPI program.
A drawback of ISP is that it does not scale for large programs due to state explosion.

Umpire applies runtime verification for MPI programs \cite{DBLP:conf/sc/VetterS00}. The approach takes one manger thread and several outfielder threads in an MPI execution. A drawback of the approach is that it relies on a concrete execution, which may miss the errors in the other execution trace.
The extensions to Umpire is Marmot \cite{DBLP:conf/parco/KrammerBMR03} and MUST \cite{DBLP:conf/ptw/HilbrichSSM09}. These approaches are neither sound nor complete for deadlock detection. 


MPI-Spin is integrated in the model checker SPIN \cite{DBLP:journals/tse/Holzmann97}, for verifying MPI programs \cite{DBLP:conf/vmcai/Siegel07,DBLP:conf/pvm/Siegel07}. It generates a model of an MPI program and symbolically executes it. It does not scale to large programs with a large degree of message non-determinism.

CIVL is a model checker which uses symbolic execution to verify a number of safety properties of various types of concurrent programs including message passing programs \cite{DBLP:conf/kbse/ZhengRLDS15,DBLP:conf/sc/SiegelZLZMEDR15}.

Vo et al. used Lamport clocks to update the auxiliary information via piggyback messages \cite{DBLP:conf/sc/VoAGSSB10,DBLP:conf/IEEEpact/VoGKSSB11}. While completeness is abandoned in their analysis, they show the work is useful and efficient in practice. 


%Elwakil et al. also used SMT techniques to reason about the program behavior in the MCAPI domain \cite{DBLP:conf/issta/ElwakilY10,DBLP:conf/atva/ElwakilYW10}. State-based and order-based encoding techniques are both used. These techniques fail to reason about the infinite buffer semantics and require a precise match set which is non-trivial to compute beforehand.

