\section{Coverage of Message Communication}

According to \algoref{algo:main}, a send and a receive from two different sections are not considered for matching. For the CTP in \figref{fig:example}, the match pair $\langle r_1\ s_1\rangle$ can not be generated with $k=1$ as the receive $r_1$ and the send $s_1$ are partitioned into two sections. 
To capture the missed match pairs for a send and a receive, a possible way is to group them into a single section with a larger bound $k$.  The match pair $\langle r_1\ s_1\rangle$ for the example in \figref{fig:example} can be generated with $k\geq2$.  
Note that all the precise match pairs for the CTP in \figref{fig:example} are generated with $k=3$, where each process is partitioned into a single section.
%\defref{def:maxbound} further states that the receives in each process and all the sends directed to that process can be grouped into a single section.
 
\begin{definition}[\textit{Max Bound}]
\label{def:maxbound}
For any CTP, there exists a bound $k$, such that each process can be partitioned into a single section; this minimal such bound $k$ is called the \textit{max bound}. 
\end{definition} 

The \textit{max bound} of $k$ can be statically computed. 
%Assuming $R_i$ is the receive list for process $p_i$, and $\textit{sender}_i$ is a set of send lists from all the potential senders to process $p_i$. 
According to \algoref{algo:main}, a section can be constructed only if each sender reaches the $k$-bound for send distribution or distributes all the sends assuming the receive list is not empty. As such, it implies that 
\begin{equation}
k= \Call{max}{\{|S(p_i)|\mid p_i \in \mathit{sender}\}},
\end{equation}
when all the sends will be grouped into a single section in the destination process, where the function $\mathrm{MAX}$ returns the maximum among a set of numbers. It is assumed that the total number of receives in a process is equal to the total number of sends directed to the process. Therefore, the process is partitioned into a single section. Further, if the bound $k$ satisfies   
\begin{equation}
k= \Call{max}{\{k_j \mid p_j\in P\}},
\end{equation} 
where $P$ is a set of all the processes in the CTP and $k_j$ is a bound that satisfies equation (5) for process $p_j$, then each process in the CTP is partitioned into a single section. The bound $k$ that satisfies equation (6) is the \textit{max bound}. 

According to the \textit{max bound} of $k$ and the match over-approximation property of the existing algorithm \cite{DBLP:conf/kbse/HuangMM13}, the precise match pairs for a CTP can be over-approximated by the algorithm in this paper.


%Since all the sends and receives are added to the sections by \algoref{algo:main} (\lemmaref{lemma:section}), the number of sends is equal to the number of receives for any section (\lemmaref{lemma:distribute}), and \textrm{MATCHAPPROX} is able to over-approximate the precise match pairs for any section (\lemmaref{lemma:match}), the only way to approximate all the precise match pairs for a CTP is to add all the receives in each process to a single section. Therefore, the proof must claim that there is exactly one section for each process, assuming $k \geq$ $\mathrm{MAX}$ $(\ceil*{\frac{\mathit{NR_{1}}}{|frm_1|}},\ldots,\ceil*{\frac{\mathit{NR_{|P|}}}{|frm_{|P|}|}})$. Applying a feasible assignment of $k$ to equation (3), it is trivial to prove that $N_k = \mathit{NR_{t}}$ for each process $p_t$. As such, all the precise match pairs for the CTP can be over-approximated.


