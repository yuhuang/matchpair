\section{Coverage of Message Communication}

As discussed earlier, a send and a receive from two different sections are not considered for matching. For the CTP in \figref{fig:example}, the match pair $\langle r_1\ s_1\rangle$ can not be generated with $k=1$ as the receive $r_1$ and the send $s_1$ are partitioned into two sections. 
To capture the missed match pairs for a send and a receive, a possible way is to group them into a single section with a larger bound $k$. The example in \figref{fig:example} also indicates that the match pair $\langle r_1\ s_1\rangle$ can be generated with $k\geq2$.  
Note that all the precise match pairs for the CTP in \figref{fig:example} are generated with $k=3$.
Further, \defref{def:maxbound} states that all the sends and receives for a process can be grouped into a single section.
 
\begin{definition}[\textit{Max Bound}]
\label{def:maxbound}
For any CTP, there exists a bound $k$, such that each process can be divided into a single section; this bound $k$ is called the \textit{max bound}. 
\end{definition} 

The \textit{max bound} of $k$ can be statically computed. 
%Assuming $R_i$ is the receive list for process $p_i$, and $\textit{sender}_i$ is a set of send lists from all the potential senders to process $p_i$. 
For any process, if $N_k$ is equal to $|R|$, only one section is divided for the process because all the receives are added to the section. 
Further, according to how $N_k$ is computed in \algoref{algo:main}, it implies that $|R|\le |\mathit{sender}|\times k$.
Therefore, the \textit{max bound} of $k$ is a number that is computed by $\mathrm{MAX}$ $(\ceil*{\frac{\mathit{|R_{1}|}}{|\mathit{sender}_1|}},\ceil*{\frac{\mathit{|R_2|}}{|\mathit{sender}_2|}},\ldots)$, where $R_i$ is the receive list for process $p_i$, and $\textit{sender}_i$ is a set of send lists from all the potential senders to process $p_i$.
The function $\mathrm{MAX}$ returns the maximum among a set of numbers.
 
\begin{theorem}
The complete set of the precise match pairs for a CTP can be over-approximated by the algorithm in this paper given the \textit{max bound} of $k$.
\label{theorem:precise}
\end{theorem}
\begin{proof}
The proof is trivial. Given the \textit{max bound} of $k$, the program has only one section for each process. Also, the existing algorithm in the prior work \cite{DBLP:conf/kbse/HuangMM13} is able to over approximate the precise match pairs for each section according to \lemmaref{lemma:match}. Therefore, all the precise match pairs for the CTP can be generated.
$\Box$

%Since all the sends and receives are added to the sections by \algoref{algo:main} (\lemmaref{lemma:section}), the number of sends is equal to the number of receives for any section (\lemmaref{lemma:distribute}), and \textrm{MATCHAPPROX} is able to over-approximate the precise match pairs for any section (\lemmaref{lemma:match}), the only way to approximate all the precise match pairs for a CTP is to add all the receives in each process to a single section. Therefore, the proof must claim that there is exactly one section for each process, assuming $k \geq$ $\mathrm{MAX}$ $(\ceil*{\frac{\mathit{NR_{1}}}{|frm_1|}},\ldots,\ceil*{\frac{\mathit{NR_{|P|}}}{|frm_{|P|}|}})$. Applying a feasible assignment of $k$ to equation (3), it is trivial to prove that $N_k = \mathit{NR_{t}}$ for each process $p_t$. As such, all the precise match pairs for the CTP can be over-approximated.

\end{proof}

