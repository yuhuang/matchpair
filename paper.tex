% This is LLNCS.DEM the demonstration file of
% the LaTeX macro package from Springer-Verlag
% for Lecture Notes in Computer Science,
% version 2.4 for LaTeX2e as of 16. April 2010
%
\documentclass[conference]{llncs}
%
\usepackage{url}
\usepackage{comment}
\usepackage{mathpartir}
\usepackage{listings}
\usepackage{alltt}
\usepackage{graphicx}
\usepackage{caption}
\usepackage{subfigure}
\usepackage{amssymb}
\usepackage{amsmath}
\usepackage{paralist, tabularx}
\usepackage{flushend}
\usepackage{multirow}
\usepackage{paralist}
\usepackage{algorithm}
\usepackage{algpseudocode}

\usepackage{pgfplots}
\pgfplotsset{width=11cm,height=8cm,compat=1.9}

\usepackage{mathtools}
\DeclarePairedDelimiter\ceil{\lceil}{\rceil}
\DeclarePairedDelimiter\floor{\lfloor}{\rfloor}


\usepackage[flushleft]{threeparttable}
\usepackage{footnote}

\makeatletter
\def\BState{\State\hskip-\ALG@thistlm}
\makeatother

\def\UrlBreaks{\do\/\do-}

\usepackage{color}
\definecolor{egmcolor}{rgb}{1.0,0.0,0.722}
\newcommand*{\egm}[1]%
%%
%% a)
{\textcolor{egmcolor}{\noindent\textbf{[egm:~}\textit{#1}]}}
%%
% Other things...
\newcommand{\figref}[1]{Figure~\ref{#1}}
\newcommand{\defref}[1]{Definition~\ref{#1}}
\newcommand{\tableref}[1]{Table~\ref{#1}}
\newcommand{\secref}[1]{Section~\ref{#1}}
\newcommand{\lemmaref}[1]{Lemma~\ref{#1}}
\newcommand{\cororef}[1]{Corollary~\ref{#1}}
\newcommand{\thmref}[1]{Theorem~\ref{#1}}
\newcommand{\algoref}[1]{Algorithm~\ref{#1}}

%\newtheorem{definition}{Definition}
%\newtheorem{theorem}{Theorem}
%\newtheorem{lemma}{Lemma}
%\newtheorem{corollary}{Corollary}

%
\begin{document}
\title{An Efficient Approach for Match Pair Approximation in Message Passing}

\author{Yu Huang \and Eric Mercer}
\institute{Brigham Young University \\
          \email{\{yuHuang,egm\}@byu.edu}}

\maketitle
%
%
\emergencystretch=1em


\begin{abstract} 
Asynchronous message passing paradigm is commonly used in high performance computing (HPC).
Message non-determinism makes the error detection in message passing programs very difficult. The prior work uses an over-approximation of the precise match pair records (each is a pair of a send and a receive that may potentially match in the runtime) to capture all possible message communication in a concurrent trace program (CTP). The SMT encoding with such a set of match pairs is able to witness program properties including deadlock, message race, and zero-buffer compatibility, however, is inefficient because of the exponential ways of match pair resolution.
This paper presents a new algorithm that under-approximates the match pairs for a CTP iteratively: first sectioning each process in the CTP such that each potential sender distributes roughly a bounded number of sends to match the same number of receives in the process, and then approximating the match pairs for the sends and receives in each section independently by a few simple rules with ranking. The algorithm runs in quadratic complexity in the number of operations. Novel in the work is that the algorithm has the flexibility to generate the match pair set with various size based on the user input. This paper further presents that the precise match pairs for any CTP can be generated with a bounded input. The benchmarks show that the algorithm in this paper drastically reduce the runtime performance of property witnessing as all the properties are witnessed with a small set of match pairs generated by the new algorithm. Experiments also show that the algorithm is able to scale to a program that employs a high degree of message non-determinism and/or a high degree of deep communication.
\end{abstract}

\keywords{Message Passing, SMT}

\newsavebox{\boxTZero}
\begin{lrbox}{\boxTZero}
\begin{minipage}[t]{0.4\linewidth}
\large
\begin{alltt}
00 \(r\sb{0}(\ast\ 0\ A\ w\sb{0})\)
01 \(w\sb{0}(r\sb{0})\)
02 \(r\sb{1}(\ast\ 0\ B\ w\sb{1})\)
03 \(w\sb{1}(r\sb{1})\)
04 \(r\sb{2}(1\ 0\ C\ w\sb{2})\)
05 \(w\sb{2}(r\sb{2})\)
06 \(s\sb{5}(0\ 1)\)
07 \(r\sb{3}(\ast\ 0\ D\ w\sb{3})\)
08 \(w\sb{3}(r\sb{3})\)
09 \(r\sb{4}(2\ 0\ E\ w\sb{4})\)
0A \(w\sb{4}(r\sb{4})\)


\end{alltt}
\end{minipage}
\end{lrbox}

\newsavebox{\boxTOne}
\begin{lrbox}{\boxTOne}
\begin{minipage}[t]{0.4\linewidth}
\large
\begin{alltt}
10 \(s\sb{0}(1\ 0)\)
11 \(s\sb{1}(1\ 0)\)
12 \(r\sb{5}(0\ 1\ F\ w\sb{5})\)
13 \(w\sb{5}(r\sb{5})\)
14 \(s\sb{2}(1\ 0)\)
\end{alltt}
\end{minipage}
\end{lrbox}

\newsavebox{\boxTTwo}
\begin{lrbox}{\boxTTwo}
\begin{minipage}[t]{0.4\linewidth}
\large
\begin{alltt}
20 \(s\sb{3}(2\ 0)\)
21 \(s\sb{4}(2\ 0)\)\end{alltt}
\end{minipage}
\end{lrbox}

% ---------------------------------------------------------------------
% END Save boxes
% ---------------------------------------------------------------------

\newcommand\examplefigone{
\begin{figure*}[tb]
\begin{center}
\setlength{\tabcolsep}{2pt}
\begin{tabular}[t]{c|c|c}
$\mathit{p_0}$ & $\mathit{p_1}$ & $\mathit{p_2}$ \\
\hline
\scalebox{0.8}{\usebox{\boxTZero}}&
\scalebox{0.8}{\usebox{\boxTOne}} &
\scalebox{0.8}{\usebox{\boxTTwo}}\\
\end{tabular}
\end{center}
\caption{A deadlock caused by orphaned receive.}
\label{fig:example}
\end{figure*}
}

\section{Introduction}
Asynchronous message passing is a widely used programming model in high performance computing (HPC). There are several common problems in message passing applications, including assertion verification, zero buffer incompatibility, and deadlock. These problems are NP--Complete for single-path programs \cite{}. This paper discusses only single-path message passing programs as they are typical of many HPC applications. 

The problems in message passing programs are difficult to detect. The major reason is that the message non-determinism that allows a receive to be match with more than one send by the runtime. Also, the program behavior is different from two semantics, infinite buffer (messages are buffered in the system) and zero buffer (no buffering in the system). Further, some message passing standard such as message passing interface (MPI) allows collective operations to synchronize a program leading to complex program behavior. 

The prior work is not efficient in reasoning about the message communication in a concurrent trace program (CTP) \cite{}. The CTP is generated from an concrete execution trace.  
In precise, the prior work uses match pair of a receive and a send to capture a potential message communication in the runtime. 
The over-approximated set of match pairs (including all the precise match pairs and may be some match pairs that never occur in the runtime) are generated to capture all possible message communications for a program in the runtime. 
The prior work then uses an SMT encoding to resolve the match pairs for error detection. However, the size of the resolution may be exponential in the number of sends and receives, therefore, makes the error detection time consuming. 

To scale well for real message passing applications, i.e., applications with large number of processes and messages, several works are proposed including dynamic analysis, runtime verification, and dynamic debugging.

The paper presents an algorithm that efficiently computes a reduced set of the precise match pairs. The set can be used as input to an SMT encoding for error checking. The number of match pair resolutions in problem solving is therefore drastically reduced leading to a scalable SMT encoding. In precise, the algorithm iteratively generates the match pairs in three steps. First, it matches a section in a CTP by queuing a sequence of receives in a common process where the number of the receives depends on a positive integer K configured by the user. It then distributes the same number of sends that may potentially match the receives from multiple processes to the section. And finally, it over-approximates the match pairs for the sends and receives in the section by comparing the ranking. The third step is inspired by the match pair generation algorithm in the prior work \cite{}. The key insight of the solution is that the problem of approximating the match pairs for all the sends and receives in a CTP can be reduced to multiple sub-problems each considering only a subset of sends and a subset of receives that are grouped as a section. A send and a receive from two different sections are not considered for matching, even though they may match in the runtime. As such, the approach is able to highly reduce the match pairs as input to an SMT encoding based on the number of sub-problems. Again, the number is dependent on the integer K. A lower value of K may generate a smaller set of match pairs. 

The paper includes the proof that the precise match pairs for any CTP can be generated with a bounded positive integer K. Also, the experiments show that the new algorithm improves the prior work in the SMT encoding such that the runtime of error detection is drastically reduced and no errors detected by the prior work are missed.


The contributions include,
\begin{compactitem}
\item the efficient algorithm that under-approximates the precise match pairs for a CTP where the size depends on a positive integer K configured by the user,
\item the proof that the precise match pairs for any CTP can be generated by the algorithm above with an appropriate positive integer K, and
\item the benchmarks demonstrate that the new algorithm improves the prior approach in the SMT encoding such that the runtime of error detection is drastically reduced and all the deadlocks in the benchmarks are detected. 
\end{compactitem}

The rest of the paper is organized as follows:

\section{Example}

\examplefigone

It is necessary to explain message communication in a simple CTP consisting of a handful of operations. Consider the CTP in \figref{fig:example} that includes three processes that use non-blocking send ($s$) and non-blocking receive ($r$) for message communication. Line numbers appear in the first column for each process. The declarations of the local variables are omitted for space. The other operations related to computation are also omitted because they are not essential to the problem in the paper.
The example shows that the behavior under infinite buffer semantics meaning that a message may be buffered in the system or in the application.

In \figref{fig:example}, the sends and receives are consumed immediately by the runtime. The send specifies two parameters where the first is the ID of the source and the second is the ID of the destination. The receive also has four parameters: the source ID, the destination ID, the local variable to store the message content, and the ID of the nearest-enclosing wait. The nearest-enclosing wait ($w$) witnesses the completion of the receive \cite{DBLP:conf/kbse/HuangMM13}. The completion of any send or receive, is only confirmed when the send or the receive is matched in the runtime. Note that if the source ID for a receive is ``$\ast$", then the receive is wildcard meaning that it may match a send from any source. 

Picking up the scenario in \figref{fig:example}, process $p_1$ receives five messages from any source or the specific sources and sends one message to $p_2$; process $p_2$ sends three messages to $p_1$ and receives a message from $p_1$; and process $p_3$ sends two messages to $p_1$. 

Given the concurrency in the scenario, a feasible schedule is a sequence of operations where the order is consistent with the order of completion in a real execution. (1) shows an instance of the feasible schedule. 
\begin{equation}
\scriptsize
s_0\rightarrow r_0 (s_0)\rightarrow s_3\rightarrow r_1 (s_3)\rightarrow s_1\rightarrow r_2 (s_1)\rightarrow 
s_5\rightarrow r_5 (s_5)\rightarrow s_2\rightarrow r_3 (s_2)\rightarrow s_4\rightarrow r_4 (s_4)
\end{equation} 
The sends and receives in (1) are totally ordered for completion. For example, the arrow between the receive $r_0$ and the send $s_3$ shows that $r_0$ is matched before $s_3$ is matched in the runtime. 
Also, the send in parenthesis for any receive shows that a message flows from the send through the receive. 
For example, the receive $r_1$ with the send $s_3$ in parenthesis indicates that the local variable $B$ in $r_1$ stores the message content from $s_3$ in the runtime. 

Since the message delivery is non-deterministic for the use of wildcard receives, there exist other feasible schedules for the concurrency in \figref{fig:example} where the messages are delivered in different ways. For example,  the receive $r_0$ can be matched with the send $s_3$ instead of the send $s_0$ if $s_3$ arrives in $p_1$ earlier than $s_0$. Given the message non-determinism, the message communication can be resolved in many (and possibly exponential) ways. 

To capture the message communication, the match pair is used for a send and a receive where they may potentially match in the runtime.
For example, the match pair $\langle r_1\ s_3\rangle$ indicates that the receive $r_1$ is potentially matched with the send $s_3$ in the runtime.
As such, the message communication for (1) can be resolved by the set of match pairs $\{\langle r_0\ s_0\rangle, \langle r_1\ s_3\rangle, \langle r_2\ s_1\rangle, \langle r_5\ s_5\rangle, \langle r_3\ s_2\rangle, \langle r_4\ s_4\rangle\}$. In the set, each send or receive is associated with a single match pair, therefore, is deterministic. Also, the message communication for any of the other schedules can be resolved by a new set of match pairs such as $\{\langle r_0\ s_3\rangle,$ $\langle r_1\ s_0\rangle,$ $\langle r_2\ s_1\rangle,$ $\langle r_5\ s_5\rangle,$ $\langle r_3\ s_2\rangle,$ $\langle r_4\ s_4\rangle\}$. As shown, the new match pairs $\langle r_0\ s_3\rangle$ and $\langle r_1\ s_0\rangle$ appear for the receives $r_0$ and $r_1$ and the sends $s_3$ and $s_0$. 

The message communication for a CTP can also be captured by collecting all the potential match pairs in a single set. This may associate a send or a receive with more than one match pair. For example, the receive $r_0$ has two match pairs $\langle r_0\ s_3\rangle$ and $\langle r_0\ s_0\rangle$ in the set. As such, the message communication for any feasible schedule for the CTP is captured by a subset of the match pairs where each send or receive is associated with a single match pair. This subset is called a resolution of the match pairs.
The SMT encoding in the prior work is able to find a resolution of the match pairs for a feasible schedule.








%\section{Overview of SMT Encoding}

%define match pair in the format of SMT encoding

%discuss how match pairs are used in SMT encoding for error detection

The crucial idea behind the prior work is encoding a CTP into an SMT problem that uses a set of formulas to constrain the program behavior based on the semantics. 
The typical errors are then checked by solving the SMT problem with additional steps. If a satisfying assignment exists for the SMT problem, then the error is detected for a feasible schedule with a resolution of the match pairs to capture the message communication in the schedule. If the SMT problem is unsatisfiable, the CTP is free of that error because no feasible schedule exists with any resolution of the match pairs.

%In precise, the violation of assertions needs to encode the negation of the tested assertions into the SMT problem \cite{}; the zero buffer incompatibility requires more restricted formulas of the zero buffer semantics encoded in the SMT problem \cite{}; and the deadlock should launch a series of static analyses to first detect a potential deadlock before validating it with the SMT problem \cite{}.  

The SMT encoding relies on the \textit{happens-before} relation in \defref{def:hb}.
\begin{definition}
The happens-before relation, denoted as $\prec_{\mathtt{HB}}$, is a partial order over operations.
\label{def:hb}
\end{definition}
Given two operations, $A$ and $B$, if $A$ must complete before $B$ in a valid program execution, then $A \prec_{\mathtt{HB}} B$ will be an SMT constraint. 
The relation is derived from the program source and potential match pairs. 

The SMT encoding specifies the constraints from the program source such that the operations in each process has to be sequentially ordered. The happens-before relation is used to define a list of rules for the constraint of sequential order. Please refer to the prior work \cite{} for the full definition of these rules. 

The SMT encoding also needs to express the message communication in the given match pairs.
Informally, a match pair equates the shared components of a send and receive and constrains the send to happen before the nearest-enclosing wait of the receive. 

\begin{definition}
A match pair, $\langle r\ s\rangle$, for a receive $r$ and a send $s$ corresponds to the constraints:
\begin{compactenum}
\item $r$ and $s$ have a common destination;
\item $r$ and $s$ have a common source, or $r$ is a wildcard receive; and
\item $s \prec_{\mathtt{HB}} nw_r$, where $nw_r$ is the nearest-enclosing wait of $r$.
\end{compactenum}
\end{definition} 
If zero buffer semantics are applied, the \textit{happens-before} relation is further constrained for any match pair $\langle r\ s \rangle$ such that $s$ and $r$ are strictly ordered, meaning that there does not exist any operation $op$ such that $s \prec_{\mathtt{HB}} op$ and $op \prec_{\mathtt{HB}} r$. 

\begin{definition}
The match pair $\langle r\ s\rangle$ is precise if the receive $r$ is matched with the send $s$ in at least one feasible schedule.
\end{definition}

%\begin{corollary}
%The bogus match pair $\langle r\ s\rangle$ implies that the receive $r$ can not be matched with the send $s$ in any feasible schedule.
%\end{corollary}

The SMT problem in the prior work is given an over-approximated set of match pairs including all the precise match pairs and maybe some unprecise match pairs. Solving the problem needs to find one among all possible match pair resolutions that satisfies the constraints in the encoding. Consider the CTPs with large number of processes and high degree of message non-determinism, the number of match pair resolutions can be exponential. Therefore, the prior work does not scale for such a CTP. This paper presents a new approach that is able to generate a smaller set of match pairs, therefore, leading to a much more efficient SMT encoding.







\section{Main Algorithm}

%Notations: 
%Process set $P$
%receive list $R(p_{dest})$
%send list $S(p_{src}, p_{dest})$

\algoref{algo:main} describes the general structure of the approach in this paper.
Intuitively, the algorithm sections each process, where each section contains a fixed number of sequential receives. It then distributes a sequence of sends from each sender process to match the receives in each section. The total number of sends is equal to the number of receives in each section. Finally, the algorithm generates the match pairs for the sends and the receives in the same section by simply comparing their ranks. The intuitive meaning of a rank is a non-negative integer that represents the position of a send or a receive in a specific sequence.

\begin{algorithm}
\caption{Main Entrance}\label{algo:main}
\begin{algorithmic}[1]
\For{$p_t\in \mathit{P}$}
%\State $N_{frm} \gets |frm_t|$
%\State $src(p)\gets\{p_1,p_2,\ldots,p_x\}$   \Comment{a set of all the potential sources for process $p$}
\While{$\mathit{NR}_{t}>0$}
\State $N_k\gets$\Call{SectionMatch}{$\mathit{k}$}
%$N_K \gets \Call{min}{|Src(p)|\times\mathit{K}, N_r(ALL,p)}$
\For{$p_f \in frm_t$}
\State $\mathit{NS_{tf}}\gets 0$
\EndFor
\State $\mathit{{NS}^\prime}\gets$\Call{DistributeSends}{$p_t$,$N_k$, $\mathit{NS}$}
\State $M\gets$\Call{MatchApprox}{$p_t$,$N_k$, $\mathit{{NS}^\prime}$}
\State \Call{remove}{$R_t$,$N_k$} 
%\State $R(p)\gets R(p)\setminus\{R(p)_i\mid 1\leq i\leq N_K\}$ 
\For{$p_{f^\prime} \in frm_t$} 
\If{$\mathit{NS}_{tf^\prime} > 0$}
\State \Call{remove}{$S_{tf^\prime}$,$\mathit{NS}_{tf^\prime}$}
\EndIf
%\State $S(p_i,p)\gets S(p_i,p)\setminus\{S(p_i,p)_j\mid 1\leq j\leq\mathit{n_s}(p_i,p)\}$
\EndFor
\EndWhile
\EndFor
\end{algorithmic}
\end{algorithm}

At a low level, the presentation needs to first explain a few data structures that are essential to the algorithm.
$\mathit{P}$ is a set of all the processes in a CTP. 
$frm_t$ is a set of the unique identifiers of all the senders for the receiver $p_t$. 
The identifier of a sender can be removed from $frm_t$ once the sender has no send to be distributed to $p_t$.
The list $R_t$ contains all the sequential receives in process $p_t$.
%$\mathit{NR}_{tf}$ is the number of the receives in $R_t$ where they can only match the sends from process $p_f$. 
%The source can be ``$\ast$" indicating the wildcard receives.
%If the source is equal to ``$A$", then 
$\mathit{NR}_{t}$ is the number of all the receives in $R_t$.
The list $S_{tf}$ contains all the sequential sends from the sender $p_f$ to the receiver $p_t$. 
$\mathit{NS}_{tf}$ is the number of the sends in $S_{tf}$ that are distributed to a specific section.

Given the data structures defined, the presentation explains \algoref{algo:main} in detail.
The algorithm iteratively checks each receive at line 2 until $R_t$ is empty.  
$\mathrm{SECTIONMATCH}$ computes the number of receives in each section at line 3. The value is assigned to $N_k$. 
$\mathrm{DISTRIBUTESENDS}$ updates the count of the distributed sends ($\mathit{NS}$) from each sender to a common section in the receiver $p_t$ at line 7. 
$\mathit{NS_{tf}}$ is initialized for each sender $p_f$ at line 4 through line 6.
$\mathrm{MATCHAPPROX}$ then stores the generated match pairs for each section in $M$ at line 8. 
Finally, the receives and the sends in each section are removed from the CTP at line 9 through line 14. 
The function $\mathrm{REMOVE}$ removes elements from the beginning of a list given two inputs: the list to remove from and the number of operations to remove. 

\subsection{Section Match}

$\mathrm{SECTIONMATCH}$ is a simple equation in (3).
\begin{equation}
\Call{min}{|frm_t|\times\mathit{k}, \mathit{NR}_{t}}
\end{equation}
The function $\mathrm{MIN}$ returns the minimum among two numbers: $|frm_t|\times\mathit{k}$ and $\mathit{NR}_{t}$.
The first number indicates that the algorithm, if possible, distributes roughly average $k$  sends from each sender and therefore, adds $|frm_t|\times\mathit{k}$ receives to a section. If there are not sufficient receives in $R_t$, then the algorithm adds all that remained in $R_t$ to the section, where the number is $\mathit{NR}_{t}$.


\begin{lemma}
\label{lemma:section}
Each receive in a CTP is added to exactly one section by \algoref{algo:main}; each send in the CTP is also added to exactly one section if $\mathrm{DISTRIBUTESENDS}$ distributes the same number of sends to match the receives in any section.
\end{lemma}
\begin{proof}
\algoref{algo:main} initializes $N_k$ the number of the receives in a single section by $\mathrm{SECTIONMATCH}$, and removes these receives from $R_t$ immediately after executing $\mathrm{MATCHAPPROXIMATE}$ at line 8. It is assumed that the same number of potential sends are distributed to the section by $\mathrm{DISTRIBUTESENDS}$ and are removed from $S_{tf^\prime}$ for each potential sender $p_{f^\prime}$ at line 12. Therefore, a receive or a send can only be added once. 
Further, \algoref{algo:main} exhaustively matches sections for all the receives in any process $p_t$ until $\mathit{NR}_{t}$ is equal to zero. 
%Also, it is assumed that the number of sends is equal to the number of receives in a CTP. 
%Each section also contains equivalent receives and sends. 
Therefore, all the receives and sends in the CTP are partitioned into sections. 
$\Box$
\end{proof}
\lemmaref{lemma:section} is used to prove the key theorem later in the paper.



\section{Send Distribution}

\algoref{algo:distribute} implements a possible way of distributing the potential sends to a specific section by computing the count of sends ($n_s$) for the common destination $p$. The algorithm is part of the function $\mathrm{DISTRIBUTESENDS}$ in \algoref{algo:main}. $N$ is the count of sends that are waiting to be distributed and is initialized to $N_K$ at line 1.The algorithm then runs in two phases. The first phase (line 2 to line 11) updates $n_s$ according to the count of the deterministic receives (the receive that has a specific source). The second phase (line 12 to line 24) updates $n_s$ by averagely distributing the sends from each potential source in $Src(p)$ to the section until $N$ is equal to zero indicating that the send distribution is finished for the section. 

\begin{algorithm}
\caption{Distribute Sends}\label{algo:distribute}
\begin{algorithmic}[1]
\State $N\gets N_K$
\For{$i\gets 1$ to $N_K$}
%\If{$r$ is a deterministic receive}
\State let $p_{r}$ be the source of the receive $R(p)_i$
\If{$p_{r}\neq\ast$}
\State $\mathit{n_s}(p_{r},p)\gets \mathit{n_s}(p_{r},p)+1$
\State $N\gets N-1$   
\If{$\mathit{n_s}(p_{r},p) = N_s(p_{r},p)$}
\State $Src(p)\gets Src(p)\setminus\{p_{r}\}$
\EndIf
\EndIf
%\EndIf
\EndFor
%\State $\mathit{avg}\gets (N_{snder}>0) ? N_{rest} / {N_{snder}} : N_{rest}$
\While{$N>0$}
\State $\mathit{avg}\gets N / {|Src(p)|}$
\For{$p_{s}\in Src(p)$}
%\If{$\mathit{n_s}(src,dest) < N_s(src,dest)$}
\If{$\mathit{n_s}(p_{s},p)<\mathit{K}\vee (\forall p_{s}^\prime\in Src(p),\mathit{n_s}(p_{s},p)\leq\mathit{n_s}(p_{s}^\prime,p))$}
\State $N_s^+ = \Call{min}{avg,N_s(p_{s},p)-\mathit{n_s}(p_{s},p)}$
\State $\mathit{n_s}(p_{s},p)\gets\mathit{n_s}(p_{s},p)+N_s^+$
\State $N\gets N-N_s^+$
\If{$\mathit{n_s}(p_{s},p) = N_s(p_{s},p)$}
\State $Src(p)\gets Src(p)\setminus\{p_{s}\}$
\EndIf
\EndIf
%\EndIf
\EndFor
\EndWhile
\end{algorithmic}
\end{algorithm}

The first phase checks each receive $R(p)_i$ in the section at line 3. If the source $p_r$ of $R(p)_i$ is not equal to ``$\ast$" at line 4 indicating that $R(p)_i$ is a deterministic receive, $n_s(p_r,p)$ is then incremented by one at line 5 and $N$ is reduced by one at line 6. The intuition is that for any deterministic receive with the source $p_r$, there has to be at least one send from $p_r$ to be distributed for the receive. If $n_s(p_r,p)$ is equal to $N_s(p_r,p)$ at line 7 indicating that all the sends from $p_r$ to $p$ are distributed, then $p_r$ is removed from $Src(p)$ at line 8 indicating that $p_r$ cannot be considered as a potential source for the destination $p$. 

The second phase iteratively distributes the remaining $N$ sends from the potential sources in $Src(p)$ until $N$ is equal to zero at line 12. 
First, the variable $avg$ is assigned to a value that is calculated by dividing $N$ by the number of the potential sources ($|Src(p)|$) at line 13. 
Each potential source $p_{s}$ is then checked at line 14. 
The condition at line 15 specifies two cases for $p_{s}$ to distribute sends. The intuition is that each potential source has to distribute an approximately average number of the sends in the two cases. The first case indicates that $n_s(p_{s},p)$ is less than the average count $K$. The second case indicates that $p_{s}$ has the lower count of sends in $n_s$ than that of any other potential source in $Src(p)$. 
If the condition at line 15 is satisfied, the variable $N_s^+$ is assigned to the minimum of the two values: $avg$ and $N_s(p_{s},p)-\mathit{n_s}(p_{s},p)$. The value indicates that the $avg$ sends are distributed from $p_s$; or, if there are not enough sends, all the remaining sends from $p_{s}$ to $p$ are distributed.  
The distribution is enforced by incrementing the value of $n_s(p_{s},p)$ and reducing the value of $N$ at line 17 and line 18 respectively. Similar to the first phase, $p_{s}$ is removed from $Src(p)$ at line 20 if all the sends from $p_{s}$ to $p$ are distributed.  

\begin{lemma}
\algoref{algo:distribute} distributes $N_K$ sends from multiple sources to a single section for matching the same number of receives in the section.
\label{lemma:distribute}
\end{lemma}


\subsection{Match Pair Approximation}

\algoref{algo:match} approximates the match pairs for the sends and receives in a specific section.
The algorithm is the part of the function $\mathrm{MATCHAPPROX}$ in \algoref{algo:main} and is inspired by the match pair generation algorithm in the prior work \cite{DBLP:conf/kbse/HuangMM13}. 
Intuitively, the algorithm checks all the pairs of receives and sends in a single section, and prunes obvious matches that cannot exist in any runtime implementation of the specification.

\begin{algorithm}
\caption{Match Approximate}\label{algo:match}
\begin{algorithmic}[1]
\State let $R_t$ be the list $r_0,r_1,\ldots,r_n$
\For{$i\gets 0$ to $N_k-1$}
\State let $p_f$ be the source of the receive $r_i$
\State let $S_{tf}$ be the list $s_0,s_1,\ldots,s_m$ for each sender $p_f$
\For{$j\gets 0$ to $\mathit{NS_{tf}}-1$}
\If{$(p_t = \ast\vee p_t = p_f)\wedge i \geq j\wedge i \leq j + (N_k - \mathit{NS_{tf}})$}
\State $M\gets M\cup\{\langle r_i\ s_j \rangle\}$
\EndIf
\EndFor
\EndFor
\end{algorithmic}
\end{algorithm}

The algorithm first checks each receive $r_i$ in a single section at line 2. The subscript $i$ represents the receive's position in $R_t$, and is named the rank of $r_i$. 
%The process $p_r$ is the source of $r_i$.
For each receive $r_i$, the algorithm checks the distributed sends from each sender $p_f$ at line 3. As only the first $\mathit{NS_{tf}}$ sends in $S_{tf}$ are distributed to the section, the subscript $j$ represents the send's position between $0$ and $\mathit{NS_{tf}}-1$, and similarly, is named the rank of $s_j$.

The condition at line 6 soundly prunes the match pairs that may never occur in the runtime with three rules. To determine if the receive $r_i$ and the send $s_j$ can be matched, the first rule requires that either $r_i$ is a wildcard receive or $r_i$ and $s_j$ matches for an identical sender. 
The second rule and the third rule constrain the ranks of $r_i$ and $s_j$, such that the messages from a common sender are received in a FIFO order. 
In precise, the second rule constrains that $i$ is greater or equal to $j$, indicating that the early messages from $p_f$ must be received by the preceding receives in $p_t$. 
The third rule also constrains the ranks of $r_i$ and $s_j$, but in a more complex structure of inequation. 
It checks whether there are sufficient sends matching the preceding receives in $p_t$. 
This number, $j + (N_k - \mathit{NS_{tf}})$, is estimated by considering the first $j$ sends from $p_f$ to $p_t$ and all the sends from other senders to $p_t$.
If the condition is satisfied, the match pair $\langle r_i\ s_j\rangle$ is added to the set $M$ at line 7. 
%Note that \algoref{algo:match} may add unprecise match pairs to $M$.

\algoref{algo:match} implies the ``match over-approximation" property in the prior work \cite{DBLP:conf/kbse/HuangMM13}. The paper presents this property in \lemmaref{lemma:match}. 

\begin{lemma}
\algoref{algo:match} over-approximates the match pairs for all the receives and sends in a specific section; that is saying, that all the precise match pairs (and maybe some unprecise match pairs) in the section are added to the output $M$.
\label{lemma:match}
\end{lemma}

%\begin{proof}
%\algoref{algo:match} must never claim to be able to prune the precise match pairs. The algorithm considers all the pairs for a send and a receive in a specific section. Also, the condition at line 4 only prunes the pairs that may never occur in the runtime: the first rule validates the endpoints consistent for the send and the receive; the second and third rules validate the FIFO order in message delivery according to the semantics. Therefore, only unprecise match pairs are pruned. $\Box$
%\end{proof}


%The condition at line 4 is then used to soundly prune the pair of $R(p)_i$ and $S(p_s,p)_j$ if they never match in the runtime with three rules. The first rule indicates that $R(p)_i$ is either a wildcard receive (the receive that may match a send from any source) or the source of $R(p)_i$ matches the source of $S(p_s,p)_j$. The second rule indicates that $i$ is greater or equal to $j$ meaning that the sends preceding $S(p_s,p)_j$ in $S(p_s,p)$ have to be matched with the receives in $p$. 
%\section{Algorithm}
\algoref{algo:main} describes the general structure of match pair generator in three steps: section match, sends distribution, and match pair approximation. 

$\mathit{P}$ is a set of processes. $Src(p)$ is a set of identifiers for the source processes that may distribute sends to match the receives on process $p$. A source process may be removed from $Src(p)$ if no send can be distributed from the process.
$R(p_{dest})$ is a list of all the receives in process $p_{dest}$. The order of the list is identical with that in the original program. $R(p_{dest})_j$ returns the $j$th. receive in the list.
The function $N_r(p_{src},p_{dest})$ returns the number of all the receives in $R(p_{dest})$ with the source $p_{src}$. If $p_{src}$ is equal to ``$ALL$", $N_r(ALL ,p_{dest})$ represents the number of all the receives in $R(p_{dest})$.
$S(p_{src},p_{dest})$ is a list of all the sends from the source $p_{src}$ to the destination $p_{dest}$. The order of the list is also identical with that in the original program. $S(p_{src},p_{dest})_i$ returns the $i$th. send in the list.
The function $N_s(p_{src},p_{dest})$ returns the number of all the sends in $S(p_{src},p_{dest})$. 

\begin{algorithm}
\caption{Main Entrance}\label{algo:main}
\begin{algorithmic}[1]
\For{$p\in \mathit{P}$}
\State $Src(p)\gets\{p_1,p_2,\ldots,p_x\}$   \Comment{a set of all the potential sources for process $p$}
\While{$N_r(ALL,p)>0$}
\State $N_K \gets \Call{min}{|Src(p)|\times\mathit{K}, N_r(ALL,p)}$
\For{$i\gets1$ to $x$}
\State $\mathit{n_s}(p_i,p)\gets 0$
\EndFor
\State $\mathit{n_s^\prime}\gets$\Call{DistributeSends}{$\mathit{n_s}$,$p$,$N_K$}
\State $M\gets$\Call{MatchApprox}{$\mathit{n_s^\prime}$,$p$,$N_K$}
\State \Call{remove}{$R(p)$,$N_K$} 
%\State $R(p)\gets R(p)\setminus\{R(p)_i\mid 1\leq i\leq N_K\}$ 
\For{$i\gets 1$ to $x$} 
\State \Call{remove}{$S(p_i,p)$,$n_s(p_i,p)$}
%\State $S(p_i,p)\gets S(p_i,p)\setminus\{S(p_i,p)_j\mid 1\leq j\leq\mathit{n_s}(p_i,p)\}$
\EndFor
\EndWhile
\EndFor
\end{algorithmic}
\end{algorithm}

The algorithm iterates over all the receives in each process $p$ until $N_r(ALL,p)$ is equal to zero at line 3 meaning that all the receives are removed from $R(p)$. 
The first step is ``section match" that matches multiple sections among the receives in $R(p)$ and all the sends in $S(p_{i},p)$ for any potential source $p_{i}\in Src(p)$ given an input $K$. 
The integer $\mathit{K}$ is used to compute the count of sends/receives ($N_K$) in each section at line 4. The function $\mathrm{MIN}$ returns the minimum of two values: $|Src(p)|\times\mathit{K}$ and $ N_r(ALL,p)$, indicating that each source in $Src(p)$ has $K$ sends in average to match the receives in $R(p)$ if possible, otherwise, the rest of the receives in $R(p)$ are all considered in the section. 

The function $\mathit{n_s}(p_{i},p)$ returns the count of sends from $p_i$ to $p$ in a specific section. The count is initialized to zero for each potential source $p_i$ at line 6 and is computed in the function $\mathrm{DISTRIBUTESENDS}$ given the input $n_s$, $p$ and $N_K$ at line 8. The new count $n_s^\prime$ is then used to approximate the match pairs for a specific section in the function $\mathrm{MATCHAPPROX}$ given two additional input $p$ and $N_K$.
The output $M$ stores the approximated match pairs. After that, the matched receives and sends in the section are removed from $R(p)$ and $S(p_i,p)$ at line 10 and line 12 respectively, indicating that they are not considered for match pair generation in the next section. The function $\mathrm{REMOVE}$ is used to remove a fixed number (the second input) of sends or receives from the beginning of a list (the first input). 
 

\begin{algorithm}
\caption{Distribute Sends}\label{algo:distribute}
\begin{algorithmic}[1]
\State $N\gets N_K$
\For{$i\gets 1$ to $N_K$}
%\If{$r$ is a deterministic receive}
\State let $p_{r}$ be the source of the receive $R(p)_i$
\If{$p_{r}\neq\ast$}
\State $\mathit{n_s}(p_{r},p)\gets \mathit{n_s}(p_{r},p)+1$
\State $N\gets N-1$   
\If{$\mathit{n_s}(p_{r},p) = N_s(p_{r},p)$}
\State $Src(p)\gets Src(p)\setminus\{p_{r}\}$
\EndIf
\EndIf
%\EndIf
\EndFor
%\State $\mathit{avg}\gets (N_{snder}>0) ? N_{rest} / {N_{snder}} : N_{rest}$
\While{$N>0$}
\State $\mathit{avg}\gets N / {|Src(p)|}$
\For{$p_{src}\in Src(p)$}
%\If{$\mathit{n_s}(src,dest) < N_s(src,dest)$}
\If{$\mathit{n_s}(p_{src},p)<\mathit{K}\vee (\forall p_{src}^\prime\in Src(p),\mathit{n_s}(p_{src},p)\leq\mathit{n_s}(p_{src}^\prime,p))$}
\State $N_s^+ = \Call{min}{avg,N_s(p_{src},p)-\mathit{n_s}(p_{src},p)}$
\State $\mathit{n_s}(p_{src},p)\gets\mathit{n_s}(p_{src},p)+N_s^+$
\State $N\gets N-N_s^+$
\If{$\mathit{n_s}(p_{src},p) = N_s(p_{src},p)$}
\State $Src(p)\gets Src(p)\setminus\{p_{src}\}$
\EndIf
\EndIf
%\EndIf
\EndFor
\EndWhile
\end{algorithmic}
\end{algorithm}


\algoref{algo:distribute} implements the function $\mathrm{DISTRIBUTESENDS}$ in \algoref{algo:main} that computes the count of sends ($n_s$) for the common destination $p$ in a specific section. $N$ is the count of sends that are not distributed and is initialized to $N_K$ at line 1.The algorithm then runs in two phases. The first phase (line 2 to line 11) updates $n_s$ based on the count of deterministic receives. The second phase (line 12 to line 24) updates $n_s$ by averagely distributing sends from any potential source in $Src(p)$ until $N$ is equal to zero. 

The first phase checks each receive in the section at line 2. If the source $p_r$ for the receive $R(p)_i$ is not equal to ``$\ast$" at line 4 meaning that $R(p)_i$ is a deterministic receive (the receive has a specific source), $n_s(p_r,p)$ is then incremented by one at line 5 and $N$ is reduced by one at line 6. The intuition is that for any deterministic receive with the source $p_r$, there has to be at least one send from $p_r$ to be distributed for the receive. If $n_s(p_r,p)$ is equal to $N_s(p_r,p)$ at line 7 indicating that all the sends from $p_r$ are distributed, then $p_r$ is removed from $Src(p)$ at line 8 indicating that it cannot be considered as a potential source for the destination $p$. 

The second phase iteratively distributes the remaining sends from any potential source in $Src(p)$ until $N$ is equal to zero at line 12. It first calculates the variable $avg$ by dividing $N$ by the number of potential sources ($|Src(p)|$) at line 13. It then checks each potential source $p_{src}$ at line 14. The condition at line 15 specifies two cases that allow $p_{src}$ to distribute sends. The intuition is that the two cases is able to distributes approximately average number of sends from each potential source. The first case indicates that $n_s(p_{src},p)$ is less than the average count $K$. The second case indicates that $p_{src}$ has lower count in $n_s$ than that of any other potential source in $Src(p)$. 
If the condition at line 15 is satisfied, the variable $N_s^+$ is assigned to the minimum of two values: $avg$ and $N_s(p_{src},p)-\mathit{n_s}(p_{src},p)$. The value indicates that the average number of sends are distributed if possible, otherwise, all the remaining sends from $p_{src}$ to $p$ are distributed.  
The distribution is enforced by incrementing $n_s(p_{src},p)$ and reducing $N$ at line 17 and line 18 respectively. Similar to the first phase, $p_{src}$ can be removed from $Src(p)$ at line 20 if all the sends from $p_{src}$ to $p$ are distributed.  


\begin{algorithm}
\caption{Match Approximate}\label{algo:match}
\begin{algorithmic}[1]
\For{$i\gets 1$ to $N_K$}
\State let $p_r$ be the source of the receive $R(p)_i$
\For{$j\gets 1$ to $\mathit{n_s}(p_s,p)$ for any possible sender $p_s$}
\If{$(p_r = \ast\vee p_r = p_s)\wedge i \geq j\wedge i \leq j + (N - \mathit{n_s}(p_s,p))$}
\State $M\gets M\cup\{\langle R(p)_i,S(p_s,p)_j \rangle\}$
\EndIf
\EndFor
\EndFor
\end{algorithmic}
\end{algorithm}

\algoref{algo:match} implements the function $\mathrm{MATCHAPPROX}$ in \algoref{algo:main} that approximates the match pairs for the receives in a common process $p$ in a specific section. The algorithm is inspired by the prior work \cite{} that uses ranks of sends and receives for match pair approximation. 

The algorithm first checks each receive with the rank from $1$ to $N_K$ in $R(p)$ at line 1. $i$ is the rank of the receive $R(p)_i$. 
For each receive, the algorithm checks each send with the rank from $1$ to $n_s(p_s,p)$ in $S(p_s,p)$ for each potential source $p_s$ at line 3.
$j$ is the rank of the send $S(p_s,p)_j$ that. The condition at line 4 then uses the same rule of rank comparison in \cite{} to soundly prune the pair $\langle R(p)_i,S(p_s,p)_j\rangle$ that may never occur in the runtime. Also, the condition checks that $R(p)_i$ is either a wildcard receive (the receive may match a send from any source) or the source of $R(p)_i$ matches the source of $S(p_s,p)_j$.
If the condition is satisfied, the pair is added to $M$ at line 5. 


%The condition at line 4 is then used to soundly prune the pair of $R(p)_i$ and $S(p_s,p)_j$ if they never match in the runtime with three rules. The first rule indicates that $R(p)_i$ is either a wildcard receive (the receive that may match a send from any source) or the source of $R(p)_i$ matches the source of $S(p_s,p)_j$. The second rule indicates that $i$ is greater or equal to $j$ meaning that the sends preceding $S(p_s,p)_j$ in $S(p_s,p)$ have to be matched with the receives in $p$. 






\input{Completeness}
\section{Experiments}
The experiments compare the performance of the SMT encoding using the new approach in this paper with that using the old match pair generation algorithm \cite{}. 
\section{Related Works}
%Rebuttal #1.4.1
The works that are directly related to our approach in this paper are those using symbolic model checking with match pairs. In particular, 
%Rebuttal
Sharma et al. proposed the first push button model checker for MCAPI -- MCC \cite{DBLP:conf/fmcad/SharmaGMH09}. It indirectly controls the MCAPI runtime to verify MCAPI programs under zero buffer semantics. An obvious drawback of the work is its inability to analyze infinite buffer semantics which is known as a common runtime environment in message passing. A key insight, though, is the direct use of match pairs.

Forejt et al. proposed a SAT based approach to detect deadlock in a single-path MPI program \cite{DBLP:conf/fm/ForejtKNS14,DBLP:journals/toplas/ForejtJKNS17}. The solution is correct and efficient for programs with a low degree of message non-determinism. However, since the size of the encoding is cubic, checking large programs is time consuming.

Elwakil et al. proposed another encoding technique that is applied to MCAPI \cite{DBLP:conf/atva/ElwakilYW10,DBLP:conf/issta/ElwakilY10}. The encoding uses a non-obvious way to constrain the happens before relation in a program. The approach fails for two reasons. First, it does not find out all possible process interleavings under infinite buffering setting. Second, the assumption about the potential matches of send and receive operations does not apply for a large complex program execution. 

An improvement of the symbolic model checking for message passing program verification is a precise SMT encoding technique that is first proposed for detecting user-provided assertions for MCAPI programs \cite{DBLP:conf/kbse/HuangMM13}. The encoding is sound and complete and is easy to use to reason about infinite buffer semantics without requiring a precise match set. %The work also provides an algorithm that runs in quadratic time complexity to generate a over-approximated match set based on the given execution trace.
This approach is then extended to checking zero buffer incompatibility for MPI semantics \cite{HuangNFM15}. 

%Rebuttal #1.4.1
The symbolic model checking above highly relies on the over-approximation of the precise match pairs which has quadratic the number of all operations in a program. The massive match pairs directly increases the difficulty and efficiency for resolving program schedules, i.e., possibly exponential more schedules have to be resolved by the SMT/SAT solver, which is extremely slow, especially for large, complex programs. 
A strength of our approach UAMP in this paper is that the input match pairs are largely shrunk by under-approximation thus the efficiency of analysis is dramatically improved while the precision in analysis is also provided.
%Rebuttal 


%A hybrid approach of static analysis and dynamic analysis also uses the SMT technique for detecting deadlocks in the pattern of orphaned receive \cite{deadlock-draft}. This approach first detects all possible potential deadlocks statically, then prunes the infeasible deadlocks by an abstracting machine with counting, and finally validate the deadlock for a feasible schedule. The validation require the SMT encoding for orpfmehaned receive deadlock. 


%Rebuttal #1.4.1
There are other solutions related to our approach in this paper. In summary, each solution is advanced for one or more aspects in analysis for message passing programs. However, our approach in this paper outperforms each of the others according to the precision and/or efficiency. 
%Rebuttal

The dynamic analyzer ISP implements the POE algorithm, a Dynamic Partial Order Reduction (DPOR) algorithm \cite{DBLP:conf/popl/FlanaganG05} applied to MPI programs \cite{DBLP:conf/ppopp/VakkalankaSGK08}. 
An extension is the MSPOE algorithm \cite{DBLP:conf/sbmf/SharmaGB12}. It operates by postponing the cooperative operations for message passing in transit until each process reaches a blocking call, and then determines the potential matches of send and receive operations in the runtime. 
%The solution is able to detect errors such as assertion violation and deadlock in an MPI program.
A drawback of ISP is that it does not scale for large programs due to state explosion.

Umpire applies runtime verification for MPI programs \cite{DBLP:conf/sc/VetterS00}. The approach takes one manger thread and several outfielder threads in an MPI execution. %A drawback of the approach is that it relies on a concrete execution, which may miss the errors in the other execution trace.
The extensions to Umpire is Marmot \cite{DBLP:conf/parco/KrammerBMR03} and MUST \cite{DBLP:conf/ptw/HilbrichSSM09}. These approaches are neither sound nor complete for deadlock detection.


MPI-Spin is integrated in the model checker SPIN \cite{DBLP:journals/tse/Holzmann97}, for verifying MPI programs \cite{DBLP:conf/vmcai/Siegel07,DBLP:conf/pvm/Siegel07}. It generates a model of an MPI program and symbolically executes it. However, it does not scale to large programs with a large degree of message non-determinism.

CIVL is a model checker which uses symbolic execution to verify a number of safety properties of various types of concurrent programs including message passing programs \cite{DBLP:conf/kbse/ZhengRLDS15,DBLP:conf/sc/SiegelZLZMEDR15}. Just like MPI-Spin, the work does not scale well for large degree of message non-determinism.

Vo et al. proposed an approach that uses Lamport clocks to update the auxiliary information via piggyback messages \cite{DBLP:conf/sc/VoAGSSB10,DBLP:conf/IEEEpact/VoGKSSB11}. The approach, however, is not complete in their analysis.

%Elwakil et al. also used SMT techniques to reason about the program behavior in the MCAPI domain \cite{DBLP:conf/issta/ElwakilY10,DBLP:conf/atva/ElwakilYW10}. State-based and order-based encoding techniques are both used. These techniques fail to reason about the infinite buffer semantics and require a precise match set which is non-trivial to compute beforehand.


\section{Conclusion and Future Work}
This paper presents a new algorithm that generates the match pairs for a message passing program iteratively. First, the algorithm matches a section by adding a sequence of receives from a common process with a positive integer K that is given as input. Second, the algorithm distributes the same number of sends from multiple processes to match those receives. Finally, the algorithm approximates the match pairs by comparing the ranks for all the potential sends and receives in the section. The key insight of the paper is that the algorithm is able to generate the match pairs for each section independently. A send and a receive from two different sections can not be considered for matching. This paper also proves that all the precise match pairs for any program can be generated with an appropriate input K. Experiments demonstrates that all the errors in the benchmarks can be efficiently detected with under-approximated match pairs generated by the new algorithm. 

A restriction of the new algorithm is that it is incapable of approximating the match pairs for sends and receives from branches. Future work will explore new approach to handle branches.




\bibliographystyle{splncs03}
\bibliography{bib/paper}
\end{document}
