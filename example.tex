

\section{Background}

\begin{comment}
%Rebuttal: Review #1.1
Message passing is widely applied in computing tasks within a single node and/or among multiple nodes.
The message passing standard such as MPI does not treat the intra-node communication and the inter-node communication differently.
In other words, sends and receives are matched with the same criteria, thus the semantics under both communications are equivalent. Note that there might be side effects for the multiple nodes to work, such as the node configuration, especially when the count of nodes reaches a very large bound. However, these effects are not related to match pair generation so is out of the scope in this paper. 
Therefore, this paper assumes a single node environment exists for all the following programs and the discussed approach can be extended to multiple nodes environment.

The runtime configuration for multi-nodes is more complex than single node, especially when the node size becomes exceptionally large. However, since the configuration is statically set up, the CTP is constructed trivially under any node environment. 
\end{comment}

%The environment of intra-node communication further needs a compete configuration file for all nodes to be launched. 
%However, the node configuration is trivial and does not interrupt with the solution in this paper. 
%Therefore, symbolic model checking with the algorithm in this paper is applicable for both communications.
%Rebuttal

\subsection{Definition of Concurrent Trace Program}

%Rebuttal #1.3
To better understand how message passing works, this paper abstracts the program model for analysis as a concurrent trace program (CTP) that is agnostic to any specific messaging platform. 
A CTP can be constructed in the runtime of single node or multi-nodes.
The configuration for multi-nodes is more complex than single node, especially when the node size becomes exceptionally large. However, since the configuration is statically set up, the CTP can be constructed trivially under both environments. 

The CTP only represents message passing and possibly collective communication. 
The message passing is inherently synchronized at points where messages are witnessed as received at a process endpoint. 
Process endpoints are purely sequential with no control flow. 
Message size is ignored to further simplify the CTP abstraction, although the unmatched size in message passing may yield error codes in runtime. However, the message passing standard stipulates that the errors do not interrupt with send-receive matches. In other words, the message size is not related to message non-determinism, which is the core problem in this paper.


\begin{figure}[h]
\begin{equation*} 
\begin{split}
&\mathrm{send:}\ s(\mathit{from}\ p_{src},\ \mathit{to}\ p_{dest},\ v)\\
&\mathrm{receive:}\ r(\mathit{from}\ p_{src},\ \mathit{to}\ p_{dest},\ \mathit{var},\ w)\\ 
&\mathrm{wait: }\ w(r)\\
&\mathrm{barrier: }\ b(\mathit{com})
\end{split}
\end{equation*}
\caption{Definition of send, receive, wait and barrier}
\label{fig:operation}
\end{figure}

\examplefigone

In general, a CTP consists of a handful of operations defined in \figref{fig:operation}, including non-blocking send ($s$), non-blocking receive ($r$), nearest-enclosing wait ($w$), and barrier ($b$). 
A send or a receive is non-blocking so its return is immediate once the operation is issued.
The parameters ``\textit{from} $p_{src}$'' and ``\textit{to} $p_{dest}$'' in any send or receive identifies the source and destination of a message respectively. The source of message in any receive may also be configured as ``$\ast$", in which case the receive is wildcard meaning a send from any source can potentially match the receive. 
The existence of wildcard receives is the direct reason for message race, or message non-determinism. 
The parameter ``$v$'' is the message content. The parameter ``\textit{var}'' is the buffer for receiving a message.

The nearest-enclosing wait is paired with a receive by the matched parameters ``\textit{w}'' and ``\textit{r}'', and witnesses the message arrival by blocking the process until the receive is actually matched with some send in execution.

Collective operations are modeled with a barrier $b$. The unique ID ``\textit{com}'' residing in the barrier operation indicates the communicator in which the process endpoint is a participant. The barrier semantics block the process endpoint until all participants in the communicator are at the barrier. Communicators can only be used one time in the CTP semantics so each interaction requires a unique ID.
%Rebuttal

%This section explains message communication in a simple CTP consisting of a handful of operations. Consider the CTP in \figref{fig:example} that includes three processes that use non-blocking send ($s$) and non-blocking receive ($r$) for message communication. The nearest-enclosing wait ($w$) witnesses the completion of the receive. The completion of any send or receive, is only confirmed when the send or the receive is matched in the runtime. 

%Note that if the sender ID for a receive is ``$\ast$", then the receive is wildcard meaning that it may match a send from any sender. 
%The declarations of the local variables are omitted for space. The other operations related to computation are also omitted because they are not essential to the algorithm in the paper.


\subsection{Example}

Picking up the scenario in \figref{fig:example} for showing the impact of message non-determinism. Process $p_1$ receives five messages from any sender or the specific senders and sends one message to $p_2$; process $p_2$ sends three messages to $p_1$ and receives a message from $p_1$; and process $p_3$ sends two messages to $p_1$. 


A set of match pairs represents a solution to the SMT problem if that set of match pairs is feasible. 
Feasible in this context means there exists a program trace allowed by the runtime that matches according to the solution set. 
%Rebuttal
A feasible match pair in the form $\langle r\ s\rangle$ indicates a receive \textit{r} and a send \textit{s} matches in execution. 
%Rebuttal
In this example, a feasible program trace is equation (1). 
%Given the concurrency in the scenario, a feasible program trace is presented in equation (1).
\begin{equation}
%\scriptsize
\begin{split}
& s_0\rightarrow r_0\rightarrow w_0\langle r_0\ s_0\rangle \rightarrow s_3\rightarrow r_1\rightarrow w_1\langle r_1\ s_3\rangle \rightarrow s_1\rightarrow r_2 \rightarrow\\
&  w_2\langle r_2\ s_1\rangle \rightarrow s_5\rightarrow r_5\rightarrow w_5\langle r_5\ s_5\rangle \rightarrow s_4\rightarrow r_3\rightarrow w_3\langle r_3\ s_4\rangle \rightarrow\\
&  s_2\rightarrow r_4\rightarrow w_4\langle r_4\ s_2\rangle
\end{split}
\end{equation} 
%In equation (1), the arrow ``$\rightarrow$" indicates the order of execution between two operations.
%The angle braces following a wait represents a match pair. For example, the match pair $\langle r_0\ s_0\rangle$ means that the send $s_0$ matches the receive $r_0$ at the wait $w_0$. 
As shown, the feasible trace in equation (1) implies a set of match pairs that represents a solution to the SMT problem from the prior work. The solution for equation (1) is $\{\langle r_0\ s_0\rangle, \langle r_1\ s_3\rangle, \langle r_2\ s_1\rangle, \langle r_5\ s_5\rangle, \langle r_3\ s_4\rangle, \langle r_4\ s_2\rangle\}$.


%Also, the sends and receives in equation (1) are totally ordered for completion. For example, the arrow between the match pair $\langle r_0\ s_0\rangle$ and the send $s_3$ shows that $r_0$ is matched before $s_3$ is matched in the runtime. 

Since message passing is possibly non-deterministic in the presence of wildcard receives, there exist other feasible executions for the same CTP where the receives are matched differently. For example,  the receive $r_0$ is matched with the send $s_3$ instead of the send $s_0$ if $s_3$ arrives in $p_1$ earlier than $s_0$. As such, the message delivery for $r_0$ is non-deterministic and it is associated with two potential match pairs $\langle r_0\ s_0\rangle$ and $\langle r_0\ s_3\rangle$. 
Given the message non-determinism, the send-receive matches may be resolved in massive ways. 
%Therefore, it is possible to capture all the message communication in a CTP only by adding all the potential match pairs in a single set. 
To capture the behavior in other executions, more match pairs are needed for analysis.
Equation (2) shows another feasible execution, in which the CTP deadlocks. 
\begin{equation}
%\scriptsize
\begin{split}
& s_0\rightarrow r_0\rightarrow w_0\langle r_0\ s_0\rangle \rightarrow s_1\rightarrow r_1\rightarrow w_1\langle r_1\ s_1\rangle \rightarrow s_3\rightarrow s_4 \\ 
& \rightarrow (\mathrm{Deadlock})
\end{split}
\end{equation} 
The deadlock occurs because there is no way to match the receives $r_2$ and $r_5$ after the sends $s_3$ and $s_4$ are issued. 
The match pair $\langle r_1\ s_1\rangle$ is essential to resolve the deadlock trace in equation (2).

Given the two execution traces above, it is evident that resolving the message non-determinism is hard as massive (probably exponential number of) possible ways of send-receive matches are needed for exploration.
The goal of the new algorithm in this paper, UAMP, is to reduce the number of all precise match pairs in the SMT solution that is still feasible for witnessing the hidden properties in a trace .

Given the example in \figref{fig:example}, UAMP would initialize $k=1$, meaning that each sender process has roughly one send distributed to a section.  Process $p_1$ is sectioned with $k=1$ as shown in equation (3).
\begin{equation} 
\begin{split}
&\mathit{section\ 1: }\{r_0,r_1,s_0,s_3\},\\
&\mathit{section\ 2: }\{r_2,r_3,s_1,s_4\},\\ 
&\mathit{section\ 3: }\{r_4,s_2\}
\end{split}
\end{equation}
Since there are two senders for $p_1$, each sender distributes one send to a section to match the same number of receives. 
For example, \textit{section 1} in equation (3) contains the receives $r_0$ and $r_1$, and the sends $s_0$ and $s_3$. 
%The algorithm in this paper then sections process $p_2$ in the same way.
Then, the match pairs for each section are over-approximated using ranks \cite{DBLP:conf/kbse/HuangMM13}. Intuitively, a rank is the position of a send or a receive in a list. For example, $r_0$ has the rank of ``0" and $r_1$ has the rank of ``1" in the receive list for \textit{section 1}. Given the ranks, the match pairs $\langle r_0\ s_0\rangle$, $\langle r_0\ s_3\rangle$, $\langle r_1\ s_0\rangle$, and $\langle r_1\ s_3\rangle$ are generated for \textit{section 1}.
The generated match pairs with $k=1$ are sufficient to capture the match solution for the trace in equation (1). 
Since the match pair $\langle r_1\ s_1\rangle$ is not generated with $k=1$, the trace in equation (2) is not captured. 

UAMP then increments $k$ to $2$ to repeat match pair generation. 
\begin{equation}
\begin{split}
&\mathit{section\ 1: }\{r_0,r_1, r_2,r_3,s_0,s_1,s_3,s_4\},\\
&\mathit{section\ 2: }\{r_4,s_2\}
\end{split}
\end{equation}
As shown in equation (4), UAMP with $k=2$ is able to partition process $p_1$ into two sections, and \textit{section 1}  groups the receive $r_1$ and the send $s_1$ at present. As such, the match pair $\langle r_1\ s_1\rangle$ is generated and the deadlock in equation (2) is witnessed.

