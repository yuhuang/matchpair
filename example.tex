\section{Concurrent Trace Program Definition and Semantics}

\examplefigone

This section explains message communication in a simple CTP consisting of a handful of operations. Consider the CTP in \figref{fig:example} that includes three processes that use non-blocking send ($s$) and non-blocking receive ($r$) for message communication. The nearest-enclosing wait ($w$) witnesses the completion of the receive \cite{DBLP:conf/kbse/HuangMM13}. The completion of any send or receive, is only confirmed when the send or the receive is matched in the runtime. Note that if the sender ID for a receive is ``$\ast$", then the receive is wildcard meaning that it may match a send from any sender. 
%The declarations of the local variables are omitted for space. The other operations related to computation are also omitted because they are not essential to the algorithm in the paper.

Picking up the scenario in \figref{fig:example}, process $p_1$ receives five messages from any sender or the specific senders and sends one message to $p_2$; process $p_2$ sends three messages to $p_1$ and receives a message from $p_1$; and process $p_3$ sends two messages to $p_1$. 


A set of match pairs represents a solution to the SMT problem if that set of match pairs is feasible. 
Feasible in this context means there exists a program trace allowed by the runtime that matches according to the solution set. In this example, a feasible program trace is equation (1). 
%Given the concurrency in the scenario, a feasible program trace is presented in equation (1).
\begin{equation}
%\scriptsize
\begin{split}
s_0\rightarrow r_0\rightarrow w_0\langle r_0\ s_0\rangle \rightarrow s_3\rightarrow r_1\rightarrow w_1\langle r_1\ s_3\rangle \rightarrow s_1\rightarrow r_2\rightarrow w_2\langle r_2\ s_1\rangle \\
\rightarrow s_5\rightarrow r_5\rightarrow w_5\langle r_5\ s_5\rangle \rightarrow s_4\rightarrow r_3\rightarrow w_3\langle r_3\ s_4\rangle \rightarrow s_2\rightarrow r_4\rightarrow w_4\langle r_4\ s_2\rangle
\end{split}
\end{equation} 
%In equation (1), the arrow ``$\rightarrow$" indicates the order of execution between two operations.
%The angle braces following a wait represents a match pair. For example, the match pair $\langle r_0\ s_0\rangle$ means that the send $s_0$ matches the receive $r_0$ at the wait $w_0$. 
As shown, the feasible trace in equation (1) implies a set of match pairs that represents a solution to the SMT problem from the prior work. The solution for equation (1) is $\{\langle r_0\ s_0\rangle, \langle r_1\ s_3\rangle, \langle r_2\ s_1\rangle, \langle r_5\ s_5\rangle, \langle r_3\ s_4\rangle, \langle r_4\ s_2\rangle\}$.


%Also, the sends and receives in equation (1) are totally ordered for completion. For example, the arrow between the match pair $\langle r_0\ s_0\rangle$ and the send $s_3$ shows that $r_0$ is matched before $s_3$ is matched in the runtime. 

Since the message matching is possibility non-deterministic in the presence of wildcard receives, there exist other feasible executions for the CTP in \figref{fig:example} where the receives are matched in different ways. For example,  the receive $r_0$ can be matched with the send $s_3$ instead of the send $s_0$ if $s_3$ arrives in $p_1$ earlier than $s_0$. As such, the message delivery for $r_0$ is non-deterministic and it is associated with two potential match pairs. 
Given the message non-determinism, the message communication can be resolved in many (and possibly exponential) ways. 
%Therefore, it is possible to capture all the message communication in a CTP only by adding all the potential match pairs in a single set. 
Therefore, to capture the behavior in other executions, new match pairs are needed.
For example, another feasible execution is presented in equation (2). As shown, the CTP in \figref{fig:example} deadlocks for this trace. 
\begin{equation}
%\scriptsize
s_0\rightarrow r_0\rightarrow w_0\langle r_0\ s_0\rangle \rightarrow s_1\rightarrow r_1\rightarrow w_1\langle r_1\ s_1\rangle \rightarrow s_3\rightarrow s_4 \rightarrow (\mathrm{Deadlock})
\end{equation} 
The deadlock occurs because there is no way to match the receives $r_2$ and $r_5$ after the sends $s_3$ and $s_4$ are issued. The use of the new match pair $\langle r_1\ s_1\rangle$ is essential to find the execution in equation (2) that makes the CTP in \figref{fig:example} deadlock.
The goal of the new algorithm in this paper is to generate a reduced set of the precise match pairs that can be used to finding such a trace with hidden errors.

Given the example in \figref{fig:example}, the algorithm in this paper would initialize $k=1$, meaning that each sender has one send distributed to a section. The algorithm sections process $p_1$ for the sequential receives.
Since there are two senders for $p_1$, each section for $p_1$ contains two receives. 
The algorithm then distributes the sends from $p_2$ and $p_3$ to each section. For example, the receives $r_0$ and $r_1$, and the sends $s_0$ and $s_3$ are grouped into a single section. Finally, the match pairs for each section are over-approximated using ranks \cite{DBLP:conf/kbse/HuangMM13}. Intuitively, a rank is the position of a send or a receive in a list. For example, $r_0$ has the rank of ``0" and $r_1$ has the rank of ``1" in the receive list for the section. By applying the simple rules given ranks, the match pairs $\langle r_0\ s_0\rangle$, $\langle r_0\ s_3\rangle$, $\langle r_1\ s_0\rangle$, and $\langle r_1\ s_3\rangle$ are generated for the section.
The generated match pairs with $k=1$ are sufficient to capture the match solution for the trace in equation (1). 
Since the match pair $\langle r_1\ s_1\rangle$ is not generated with $k=1$, the trace in equation (2) is not captured. 
The algorithm then raises $k$ to $2$ to repeat the under-approximation of match pairs. At present, the receive $r_1$ and the send $s_1$ are grouped into a single section, and the match pair $\langle r_1\ s_1\rangle$ is generated.

